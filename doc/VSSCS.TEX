\documentstyle[12pt,fullpage]{article}
\begin{document}
\noindent i switched from SCCS to RCS many moons ago. the set of facilities that
RCS provides is a proper superset of the facilities provided by SCCS.
in my opinion key advantages of RCS over SCCS are:
\begin{itemize}
\item in addition to providing version control, RCS provides very flexible
  configuration management facilities. ({\bf configuration}: set of versions
  of files that make up a release/snapshot of a software system). it provides
  both selection and composition mechanisms for configuration management. RCS
can manage multiple configurations.
\item provides facilities for merging updates from released configurations
into the main development branch {\bf and} vice--versa.
\item provides automatic identification. ({\bf identification}: stamping
revisions and configurations with unique markers that unambiguously identifies
the configuration -- and recursively so.)
\item can control access control to be strict or loose.
\item can provide access control beyond unix file protections.
\item provides very flexible tools for pruning and maintaining an ancestral tree.
\item provides user definable states to manage conflicting updates.
\item tools (tried and proven) exist to manage multiple ancestral trees.
\item tools exist to manage multiple directory hierarchies that
make up a software system.
\item RCS software runs on multiple platforms, including lots of non--unix ones.
\item RCS's delta storage method is both more space and time efficient than
SCCS's. (reverse deltas as opposed to interleaved deltas).
\item its much better integrated with {\em make}(1).
\end{itemize}
\noindent if you need further info, documentation, references, statistics, or
just want to shoot the breeze please drop by.\\
cheers, bammi
\end{document}
